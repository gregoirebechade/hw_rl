\input{preamble}
\input{format}
\input{commands}

\begin{document}

\begin{Large}
    \textsf{\textbf{Stochastic Linear Bandits}}
    An Empirical Study
\end{Large}

\vspace{1ex}

\textsf{\textbf{Students:}} \text{Alexis Marouani, Grégoire Béchade}, \\
\textsf{\textbf{Lecturer:}} \text{Claire Vernade}, Contact me on Slack if anything looks weird, or find my email on my \href{www.cvernade.com}{website} 

\section{Problem 1: Linear epsilon greedy}

\begin{enumerate}
    \item q1
    \item q2
    \item According to the documentation of numpy, the complexity of the pinv function is $O(min(n m^2, n^2m))$. In our problem, the matrix is squared, of size $d$ so the complexity is $O(d^3)$.
This can create problems when facing high-dimensional problems. We have therefore decided to implement a class LinearEpsilonGreedybis, in which we have changed the estimation of $\hat{theta}$. 
Instead of estimating $\theta$ through the least square estimator, we decided to estimate it through this estimator: $\hat{\theta} = \sum_{t=1}^{T} \left\langle \theta , A_t\right\rangle A_t$. 
We didn't manage to find theoretical guarantees about the expected value of this estimator, as $\mathbb{E}(\hat{\theta}) = \sum_{t=1}^{T} \mathbb{E} ( \left\langle \theta , A_t\right\rangle A_t) $, which can't be precised without assumptions on the distribution of $A_t$.
However, we have tested it on different problems, and it seems to obtain the same results as the one obtained with the least square estimator.
Computing $\hat{\theta}$ has a complexity in $0(d)$, as we only have to compute scalars products of d-vectors. The figure \ref{fig:lin_epsilon_greedy} underlines the gain in computational time, while the performances are the same.

\begin{figure}[h]
    \centering
    \includegraphics[width=0.5\textwidth]{images/comparison.png}
    \caption{Comparison of the performances and rime of execution of LinearEpsilonGreedy and the LinearEpsilonGreedy bis, with N=50, T=200 and 20 tries.} 
    \label{fig:lin_epsilon_greedy}
\end{figure}





\end{enumerate}


\section{Problem 2: LinUCB and LinTS}


\begin{enumerate}
    \item q1
    \item q2
    \item q3
\end{enumerate}

\end{document}